%    BytesLibres.tex is free software: you can redistribute it and/or modify
%    it under the terms of the GNU General Public License as published by
%    the Free Software Foundation, either version 3 of the License, or
%    (at your option) any later version.
%
%    BytesLibres.tex is distributed in the hope that it will be useful,
%    but WITHOUT ANY WARRANTY; without even the implied warranty of
%    MERCHANTABILITY or FITNESS FOR A PARTICULAR PURPOSE.  See the
%    GNU General Public License for more details.
%
%    You should have received a copy of the GNU General Public License
%    along with BytesLibres.tex  If not, see <http://www.gnu.org/licenses/>.

% % % % % % % % % % % % % % % % % % % % % % % % % % % % % % % % % % % % % % % % 
%    Name: 		BytesLibres.tex
%    Date:  		12-January-2014
%    Author:  		jm@taquinux.com
% % % % % % % % % % % % % % % % % % % % % % % % % % % % % % % % % % % % % % % % 

\documentclass[12pt]{report}

% LaTeX Packages
\usepackage{amsmath}										% Math symbols
\usepackage{setspace}									% Interline space
\usepackage[normalem]{ulem}							% Underline options
\usepackage{longtable}									% Tables that span more than a page
\usepackage[usenames, dvipsnames]{color}			% Named colors
\usepackage[utf8x]{inputenc}							% Spanish characters
\usepackage{graphics}									% Add support for additional image formats
\usepackage[pdftex]{graphicx}							% Add support for additional image formats
\usepackage[OT1]{fontenc}								% Add support for additional fonts
\usepackage{multicol}									% Multi-column
\usepackage{pifont}										% Symbols
\usepackage{lastpage}									% Page counter
\usepackage{caption}										% Change format of 'caption'
\usepackage{natbib}
\usepackage{hyperref}									% Redefine links
\usepackage{listing}										% Formating the code command
\usepackage{eso-pic}										% Fullpage images
\usepackage{xparse}										% Command overloading
\hypersetup
{
	pdftex,%
	colorlinks=true,%
	linkcolor=black,%
	urlcolor=blue,%
	citecolor=blue,
	filecolor=blue											% Change link's color
}
\usepackage[table]{xcolor}								% Change color on tables
\usepackage{titlesec}									% Change font for sections
\usepackage[spanish]{babel}							% Change languaje to spanish

% % % TABLE OF CONTENTS
\makeatletter
\renewcommand\tableofcontents[2]{
	\begin{minipage}{0.9\columnwidth}
		\begin{center}
		\includegraphics[height=#2]{#1}\\
		\normalfont
		\end{center}
	\end{minipage}		

	\vspace{-#2}
	\vspace{2.0cm}

	\normalfont
	\hspace{.4\textwidth}								% Initial space for text
	\vspace{-1.25cm}
	\begin{spacing}{0.85}

		\color{titleBarColor} \centering \rule{0.5\textwidth}{2pt} \vspace{0.0cm}
            
		% Title
		\vspace{0.5cm}
		\parbox[t]{0.75 \textwidth} { \centering \fontsize{38}{0} \usefont{OT1}{phv}{b}{n} \color{titleTextColor}{Contenido}% 				
      								\vspace{0.5cm}}
		\normalfont
		\color{titleBarColor} \centering \rule{0.5\textwidth}{2pt}
		\normalfont         
	\end{spacing}
	\textnormal{}
	\\*

	\begin{multicols}{2}        
	\@starttoc{toc}
	\end{multicols}
	
}
\makeatother

% % % PAGE SIZE AND BORDERS
\usepackage[top=2cm, bottom=2cm, left=1.5cm, right=1.5cm, paperwidth=8.5in, paperheight=11in]{geometry}

\renewcommand*\familydefault{\sfdefault} %% Only if the base font of the document is to be sans serif

% % % Numbers for index
% Numeration depth in the document
\setcounter{secnumdepth}{-1}
% Numeration depth in the table of contents
\setcounter{tocdepth}{3}	

\usepackage{alltt}									% Monospace text on 'code'
\usepackage{ragged2e}								% Enable justified text
\usepackage[normalem]{ulem}						% Tachado de texto

\usepackage{natbib}
\usepackage{tikz}																	% Drawings
\usetikzlibrary{positioning,decorations.pathreplacing}				% Symbols for diagrams
\usetikzlibrary{shapes,arrows}												% Arrows for diagrams
% State diagram blocks
% Define block styles
\tikzstyle{decision} = [diamond, draw, fill=blue!20, 						
    text width=4.5em, text badly centered, node distance=3cm, inner sep=0pt]
\tikzstyle{block} = [rectangle, draw, fill=blue!20, 
    text width=5em, text centered, rounded corners, minimum height=4em]
\tikzstyle{line} = [draw, -latex']
\tikzstyle{cloud} = [draw, ellipse,fill=red!20, node distance=3cm,
    minimum height=2em]

\captionsetup[figure]{labelfont=bf,textfont=normalfont}

% % % FRONT PAGE CONFIGURATION
\definecolor{TitleColor3}{RGB}{0,0,0}
\definecolor{TitleColor2}{RGB}{0,0,127}
\definecolor{TitleColor1}{RGB}{0,0,255}

% 1 - Title
% 2 - Second title
% 3 - Main title
% 4 - Author A
% 5 - Author B
% 6 - Date
\newcommand{\BytesLibres}[6]
{
\begin{titlepage}
	\begin{center}
		\begin{minipage}{0.95\textwidth}
		
			\begin{center}
			%        Separation between image and border
			\rule{\linewidth}{0.0mm} \\[1.5cm]
			
			% Title
         \textsc{\fontsize{50}{0} \textbf{ \color{TitleColor1}{ \texttt{#1}}} }\\[-0.8cm]
         \textsc{\fontsize{50}{0} \textbf{ \color{TitleColor2}{ \texttt{#1}}} }\\[-0.8cm]
         \textsc{\fontsize{50}{0} \textbf{ \color{TitleColor3}{ \texttt{#1}}} }\\
			% Second title
			\textsc{\Large{#2}}\\[2.5cm]


			% Image
			\includegraphics[width=0.5\textwidth]{Tux.png}	\\[-10.0cm]

         % Main article
         \begin{flushright}
			{ \huge \bfseries {#3}}         
         \end{flushright}
			
			% Content
			\begin{minipage}
   			{0.8\textwidth}
   			\begin{flushleft}
   			\large {#4}
   			\end{flushleft}
			\end{minipage}
			
			% End of the page
			% City and Date
			\begin{minipage}{0.7\textwidth}
			      \begin{center} \large {#5} \end{center}			      
			\end{minipage}
			% Property
			\begin{minipage}{0.8\textwidth}
			   \begin{center} \large {#6} \end{center}
			\end{minipage}
			
			\end{center}		
		\end{minipage}	
	\end{center}
\end{titlepage}
}

\newcommand{\BytesLibresPortada}[1]
{
	\begin{titlepage}
		% Put something to make the page break work
		\begin{flushright}
			{ \huge \bfseries {}}         
		\end{flushright}
	
		\AddToShipoutPictureBG*
		{
		    \put(-4,0)
		    {
		        \parbox[b][\paperheight]{\paperwidth}
		        {
		            \vfill
		            \centering
		            \includegraphics[width=\paperwidth,height=\paperheight]{Portada.jpg}
		            \vfill
		         }
		    }
		}
	\end{titlepage}
	\pagebreak[4]
}

\newcommand{\FigurePage}[1]
{   
   \pagebreak[4]
	% Put something to make the page break work
	\begin{flushright}
		{ \huge \bfseries {}}         
	\end{flushright}

   \thispagestyle{empty}
   
	\AddToShipoutPictureBG*
	{
	   \thispagestyle{plain}
	    \put(-4,0)
	    {
	        \parbox[b][\paperheight]{\paperwidth}
	        {
	            \vfill
	            \centering
	            \includegraphics[width=\paperwidth,height=\paperheight]{Portada.jpg}
	            \vfill
	         }
	    }
	}
	\pagebreak[4]
}


% % % CHAPTER STUFF
\addto\captionsspanish{\renewcommand{\chaptername}{GULAG}}

% Colors
\definecolor{titleTextColor}{RGB}{139,0,0}
\definecolor{titleBarColor}{RGB}{184,134,11}
\definecolor{titleAbstract}{RGB}{0,0,0}
\definecolor{titleAuthor}{RGB}{0,0,128}

% Redefine the command
\makeatletter									% Enable the use of @ has a letter
\renewcommand{\@makechapterhead}[1]
{
      % Do nothing \Chapter is used instead
}
\normalfont																			% Reset font
\makeatother																		% Disable '@' as character

% \Chapter{Text for ToC}{Text for Article}{Abstract}{Author}
\DeclareDocumentCommand \Chapter {m m m m g}%
{
   \chapter[#2]{#2}						% Title creation for ToC   
   \normalfont
   \hspace{.4\textwidth}				% Initial space for text
   \vspace{-1.25cm}
   \begin{spacing}{0.85}

      \color{titleBarColor} \centering \rule{0.5\textwidth}{2pt} \vspace{0.0cm}
            
      % Title
      \vspace{0.5cm}
      \IfValueT {#5} {\includegraphics[height=5.0cm]{#5}}
      
      \parbox[t]{0.75 \textwidth} { \centering \fontsize{38}{0} \usefont{OT1}{phv}{b}{n} \color{titleTextColor}{#1} }
      \normalfont
      % Abstract
      \parbox[t]{0.85 \textwidth} { \centering \fontsize{14}{0} \usefont{OT1}{phv}{m}{it} \color{titleAbstract}{#3} }
      \normalfont
      % Author
      \parbox[t]{0.95 \textwidth} { \raggedleft \fontsize{12}{0} \usefont{OT1}{phv}{m}{n} \color{titleAuthor}{#4} 													\vspace{0.5cm}}
      \normalfont
      \color{titleBarColor} \centering \rule{0.5\textwidth}{2pt}
      \normalfont         
   \end{spacing}
   \textnormal{}
}

% % % SECTION STUFF
\titleformat{\section}{\bf\normalsize\bfseries}{\thesection}{10em}{}{}{}

% % % REMARK STUFF
\definecolor{RemarkTop}{RGB}{205,133,63}
\definecolor{RemarkBody}{RGB}{255,215,0}
\newcommand{\Remark}[2]
{
	\begin{center}
		\fcolorbox{RemarkTop}{RemarkTop}
		{
			\begin{minipage}{0.9\columnwidth}
				\Large 		\bf \textcolor{white}{\ding{228}}
				\large \bf \textcolor{white}{#1}
			\end{minipage}	
		}
		\\*
		\vspace{-1pt}
		\fcolorbox{RemarkTop}{RemarkBody}
		{
			\begin{minipage}{0.9\columnwidth}
				\textcolor{black}{#2}
			\end{minipage}		
		\normalfont
		}
	\end{center}	
}


% % % CODE STUFF
\definecolor{CodeTop}{RGB}{192,192,192}
\definecolor{CodeBody}{RGB}{211,211,211}
\newcommand{\Code}[2]
{
	\begin{center}
		\fcolorbox{CodeTop}{CodeTop}
		{
			\begin{minipage}{0.9\columnwidth}
				\Large 		\bf \textcolor{white}{\ding{46}}
				\large \bf \textcolor{white}{#1}
			\end{minipage}	
		}
		\\*
		\vspace{-1pt}
		\fcolorbox{CodeTop}{CodeBody}
		{
			\begin{minipage}{0.9\columnwidth}	   
				\textcolor{black}
				{\fontsize{10}{0}\texttt{#2}}
			\end{minipage}		
		\normalfont
		}
	\end{center}	
}


% % % FIGURES STUFF
% % % Colors
\definecolor{fig_b}{RGB}{160,160,160}
\definecolor{fig_f}{RGB}{240,240,240}

% % % Command
\newcommand{\Figure}[5]
{
   \begin{center}
	\fcolorbox{fig_b}{fig_b}
		{
			\begin{minipage}{0.9\columnwidth}
				\color{black}
				\Large 	\bf 	{\ding{45}}
				\large	\bf 	{#3}
			\end{minipage}	
		}
		\\*
		\vspace{-1pt}
		\fcolorbox{fig_b}{fig_f}
		{
			\begin{minipage}{0.9\columnwidth}
				\begin{center}
				\includegraphics[width=#1]{#2}\\
				\normalfont
				{#4}
				\end{center}
			\end{minipage}		
		\normalfont
		}
   \end{center}	
}


% % % TABLE STUFF
\definecolor{TableColor1}{RGB}{205,133,63}
\definecolor{TableColor2}{RGB}{255,215,0}


% % % % % HEADERS AND FOOTERS
% % % FOOT NOTES
\usepackage{perpage} 								% The perpage package
\MakePerPage{footnote} 								% The perpage package command
\renewcommand{\thefootnote}{\roman{footnote}}
\usepackage{endnotes}
\renewcommand{\notesname}{Referencias:}
\interfootnotelinepenalty=10000

\makeatletter
\def\enoteheading{%
	\section*{%
		\notesname
		\@mkboth{\chaptername}{}
	}%
	\mbox{}\par\vskip-\baselineskip
	}
\makeatother 


\usepackage{fancyhdr}

\setlength{\topmargin}{-1.0cm}						% Border of header
\setlength{\textheight}{21.0cm}						% Font size
\setlength{\footskip}{2.0cm}							% Border of footer
\pagestyle{fancyplain}
\fancypagestyle{fancy}{
	\fancyhead[LE]{\chaptername}
}
\renewcommand{\headrulewidth}{1.0pt}				% Header bar
\renewcommand{\footrulewidth}{1.0pt}				% Footer bar

% % % Remove bar above footnotes
\renewcommand{\footnoterule}{%
  \kern -3pt

  \kern 2pt
}

\cfoot{\thepage\ de \pageref{LastPage}}       % Center footer

\usepackage{rotating}									% Rotate text in tables
